\documentclass[a4paper,11pt]{report}
\usepackage[T1]{fontenc}
\usepackage[utf8]{inputenc}
\usepackage{lmodern}
\usepackage[spanish]{babel}
\usepackage[dvips]{graphics,color,epsfig}
%\usepackage[latin1]{inputenc}

\usepackage{pst-all}
\usepackage{pstricks}

\usepackage{amssymb} % Math
\usepackage{graphicx}
\usepackage{float}
\usepackage{amsmath}
%\input{definiciones_diego}
\usepackage{comment}
\usepackage{physics}

\begin{document}
\begin{center}
{\Large Abstract AGU Fall Meeting 2020}\\
\end{center}
Authors: D.G. Lloveras(1), A.M. Vásquez(1), F.A. Nuevo(1), N. Sachdeva(2), W. Manchester IV(2), B. Van der Holst(2), P. Lamy(3), J. Wojak(3).\\
\\
{\small
(1) IAFE (UBA-CONICET), Ciudad Autonoma de Buenos Aires, Argentina. \\
(2) CLaSP (Univ. of Michigan), Ann Arbor, Michigan, EEUU.}\\
(3) Laboratoire d’Astrophysique de Marseille, Aix-Marseille Université, Marseille, France.\\

{Title: Comparing magnetohydrodynamic AWSoM model with DEMT reconstruction during WHPI}\\

In the current space-age, accurate three-dimensional (3D) magnetohydrodynamic (MHD) models are required in order to predict space weather. To achieve this, the models must first be verified with observations and must be validated. In this context, a new minimum of solar activity presents an opportunity to observe, understand, and model the solar corona in its simplest conditions. We carry out a detailed comparison and validation of the Alfvén Wave Solar Model (AWSoM) within the Space Weather Modeling Framework (SWMF) using the Differential Emission Measure Tomography (DEMT). Using EUV narrowband images taken by SDO/AIA, the semi-empirical technique DEMT provides 3D reconstruction of coronal electron density and temperature in the low corona (1.0-1.25 Rsun). \textcolor{red}{Using white light images taken by SOHO/LASCO-C2, DEMT provides 3D electron density reconstructions (2.55 <r< 6 Rsun).} To validate AWSoM with tomographic results, we selected two Carrington rotations (CR) of specific interest to the Whole Heliosphere and Planetary Interactions (WHPI). On one hand, CR 2219 related to the total solar eclipse campaign of 2019 and on the other hand, CR 2223 related to the Parker Solar Probe 4th perihelion campaign. We discuss the capability of AWSoM to model the observations in different magnetic structures.

\end{document}





