\documentclass[]{article}

\usepackage[utf8]{inputenc} % permite utilizar caracteres internacionales.
\usepackage[dvipsnames]{xcolor} % Introduce la gama de colores listados con ejemplos en: https://www.overleaf.com/learn/latex/Using_colours_in_LaTeX
\pagestyle{empty} % no numerar páginas.

\begin{document}

\centerline{Abstract AGU Fall Meeting 2020}

\vskip 0.5cm
\noindent{\bf Three-Dimensional Tomographic Reconstruction and MHD Modeling of WHPI target rotations CR-2219 and CR-2223}

\vskip 0.5cm
\noindent
D.G. Lloveras(1), A.M. Vásquez(1), F.A. Nuevo(1), N. Sachdeva(2), W. Manchester IV(2), B. Van der Holst(2), R. Frazin(2), P. Lamy(3) and J. Wojak(3).

\vskip 0.5cm
\noindent
(1) Instituto de Astronomía y Física del Espacio (UBA-CONICET), Ciudad Autónoma de Buenos Aires, Argentina\\
(2) Climate and Space Sciences and Engineering, University of Michigan, Ann Arbor, Michigan, USA\\
(3) Laboratoire d’Astrophysique de Marseille, Aix-Marseille Université, Marseille, France\\


\vskip 0.5cm
\noindent
In the current space age, accurate prediction of space weather conditions requires state-of-the-art three-dimensional (3D) magnetohydrodynamic (MHD) models, which need to be validated with observational data. The recent deep minimum of solar activity, between solar cycles 24 and 25, renews the opportunity to study the Sun-Earth connection under the simplest solar and space environmental conditions. The international Whole Heliosphere and Planetary Interactions (WHPI) initiative aims at this specific purpose. In this work, we study two WHPI campaign periods, the July 2019 total solar eclipse Carrington rotations (CR)-2019, and the Parker Solar Probe and STEREO-A closest approach CR-2223. Based on narrowband EUV data provided by the SDO/AIA instrument we carry out tomographic reconstruction of the coronal electron density and temperature in the range of heliocentric heights $r < 1.25~{\rm R}_\odot$. Based on visible light coronagraph data provided by the SoHO/LASCO-C2 instrument we carry out tomographic reconstruction of the coronal electron density and in the range of heliocentric heights $2.5-6.0~{\rm R}_\odot$. Based on ADAPT-GONG synoptic magnetograms we compute 3D-MHD models using the Alfvén Wave Solar Model (AWSoM) within the Space Weather Modeling Framework (SWMF). We study the capability of the model to reproduce the tomographic reconstructions in different coronal magnetic structures, both open and closed. In particular, we investigate the relationship between the 3D distribution of the physical parameters of the terminal solar wind of the model and the 3D tomographic reconstruction of the thermodynamical properties in the low corona.

\end{document}



