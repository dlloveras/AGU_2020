\documentclass[]{article}

\usepackage{amssymb}
\usepackage[utf8]{inputenc} % permite utilizar caracteres internacionales.
\usepackage[dvipsnames]{xcolor} % Introduce la gama de colores listados con ejemplos en: https://www.overleaf.com/learn/latex/Using_colours_in_LaTeX
\pagestyle{empty} % no numerar páginas.

\def\edit#1{\textcolor{red}{#1}}


\begin{document}

\centerline{Abstract AGU Fall Meeting 2020}

\vskip 0.5cm
\noindent{\bf Three-Dimensional Tomographic Reconstruction and MHD Modeling of WHPI target rotations CR-2219 and CR-2223}

\vskip 0.5cm
\noindent
D.G. Lloveras(1), A.M. Vásquez(1), F.A. Nuevo(1), N. Sachdeva(2), W. Manchester IV(2), B. Van der Holst(2), R. Frazin(2), P. Lamy(3) and J. Wojak(4).

\vskip 0.5cm
\noindent
(1) Instituto de Astronomía y Física del Espacio (UBA-CONICET), Ciudad Autonoma de Buenos Aires, Argentina\\
(2) Climate and Space Sciences and Engineering, University of Michigan, Ann Arbor, Michigan, USA\\
(3) Laboratoire Atmosph\`eres, Milieux et Observations Spatiales, CNRS \& UVSQ, Guyancourt, France\\
(4) Aix Marseille Universit\'e, CNRS, Centrale Marseille, Institut Fresnel, Marseille, France\\


\vskip 0.5cm
\noindent
Accurate prediction of space weather conditions requires state-of-the-art three-dimensional (3D) magnetohydrodynamic (MHD) models, which need to be validated with observational data. The recent deep minimum of solar activity, between solar cycles 24 and 25, renews the opportunity to study the Sun-Earth connection under the simplest solar and space environmental conditions. The international Whole Heliosphere and Planetary Interactions (WHPI) initiative aims at this specific purpose. In this work, we study two WHPI campaign periods, the July 2019 total solar eclipse Carrington rotation (CR)-2019, and the Parker Solar Probe and STEREO-A closest approach CR-2223. Based on narrowband EUV data provided by the SDO/AIA instrument we carry out tomographic reconstruction of the coronal electron density and temperature in the range of heliocentric heights $r \lesssim 1.25~{\rm R}_\odot$. Based on visible light coronagraph data provided by the SoHO/LASCO-C2 instrument we carry out tomographic reconstruction of the coronal electron density and in the range of heliocentric heights $\approx 2.5-6.0~{\rm R}_\odot$. Applying ADAPT-GONG synoptic magnetograms as boundary conditions, we use the Alfven Wave Solar Model (AWSoM) to simulate the corona and solar wind for these time periods. We study the capability of the 3D-MHD model to reproduce the tomographic reconstructions {in both closed and open coronal magnetic structures. In coronal holes in particular, we investigate the correlation between the reconstructed 3D distribution of the thermodynamical properties in the low corona and the 3D distribution of the physical parameters of the terminal solar wind of the model, discriminating its fast and slow components.

\end{document}



